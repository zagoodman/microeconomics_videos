\begin{spacing}{1.0} 
\begin{table} \centering \caption{Spillover Effects of Incentive on Other Studying} 
\label{spillover_studying} 
\begin{threeparttable} 
\begin{tabular}{m{0.35\linewidth} *{5}{>{\centering\arraybackslash}m{0.1\linewidth}}}
\toprule
                               & Control Mean &     (1) &     (2) &     (3) &     (4) \\
\midrule
             Attendance checks &         5.91 &   -0.08 &   -0.09 &   -0.16 &   -0.10 \\
                               &              &  (0.18) &  (0.17) &  (0.17) &  (0.18) \\
             Num. Piazza views &        49.81 &   10.64 &    8.51 &   10.64 &    3.69 \\
                               &              &  (7.64) &  (8.25) &  (7.60) &  (8.05) \\
       Num. Piazza days online &        10.40 &    1.43 &    1.89 &    1.43 &    1.67 \\
                               &              &  (1.55) &  (1.59) &  (1.54) &  (1.65) \\
   Num. Piazza questions asked &         0.53 &    0.32 &    0.30 &    0.32 &    0.30 \\
                               &              &  (0.25) &  (0.30) &  (0.25) &  (0.31) \\
           Num. Piazza answers &         0.47 &    0.08 &    0.01 &    0.08 &   -0.02 \\
                               &              &  (0.26) &  (0.28) &  (0.26) &  (0.28) \\
           Num. of PSET visits &         0.41 &    0.05 &   -0.01 &    0.07 &    0.00 \\
                               &              &  (0.13) &  (0.14) &  (0.12) &  (0.12) \\
                  \midrule 
 Observations &              &     374 &     332 &     374 &     332 \\
 Treatment assignment controls &              &     Yes &      No &     Yes &     Yes \\
          Demographic controls &              &      No &      No &     Yes &     Yes \\
            Pair Fixed Effects &              &      No &      No &      No &     Yes \\
\bottomrule
\end{tabular}
\Fignote{This table reports coefficients on $Incentive_i$ from Equations \ref{itt_spec} and TBD. Model (1) contains linear controls midterm 1 score and year; (2) is the difference in means and standard errors calculated using the repeated sampling framework of Neyman (1923); (3) and (4) use the post-double-selection (PDS) procedure of \textcite{bch2014a} to select control variables then estimate treatment effects and standard errors. The control variables selected using PDS are listed in Table \ref{controlvars_selected_itt}. Models (2) and (4) contain only students whose matched-pair did not attrite from the experiment. There were seven \textit{Attendance checks} during the quarter. \textit{PSET visits} includes those after the first midterm. \textit{Control Mean} is the mean for the Control students included in models (1) and (3), which is nearly identical to the mean for the Control students included in models (2) and (4). \Regnote} 
\end{threeparttable}
\end{table} 
\end{spacing}